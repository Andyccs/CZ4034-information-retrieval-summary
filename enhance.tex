\chapter{Enhancing Retrieval}

\begin{multicols*}{2}

\noindent Relevance is assessed relative to the user need not the query.

\begin{itemize}
    \item Precision: fraction of retrieved documents that are relevant
    \item Recall: fraction of relevant document that are retrieved
    \item Conservatism: high precision, low recall
    \item Liberalism: low precision, high recall
\end{itemize}

\begin{center}
\begin{tabular}{ |c|c c| } 
    \hline
     & Relevant & Non relevant \\
    \hline 
    Retrieved & TP & FP \\
    Not retrieved & FN & TN \\
    \hline
\end{tabular}
\end{center}

$$P=\frac{TP}{TP+FP}$$
$$R=\frac{TP}{TP+FN}$$

\noindent F-measure is weighted harmonic mean of Precision and Recall:

\noindent where:
\begin{equation*}
\begin{split}
    F &=\frac{1}{\alpha \frac{1}{P} + (1-\alpha)\frac{1}{R}} \\
    &= \frac{2PR}{P+R}, \text{ when } \alpha = 0.5
\end{split}
\end{equation*}

\section{Rank-based Measures}
\noindent We will use the following example for Precision@K, Recall@K and MAP:

\noindent \verb|relevant, not, relevant, not, relevant|

\subsection{Precision@K and Recall@K}

$$\text{Precision@3} = 2 / 3$$
$$\text{Precision@4} = 2 / 4$$
$$\text{Precision@5} = 3 / 5$$

\subsection{Mean Average Precision (MAP)}

$$\text{AveragePrecision} = \text{average of P@K}$$
$$\text{AveragePrecision} = \frac{1}{3} \cdot \Big(\frac{1}{1} + \frac{2}{3} + \frac{3}{5} \Big)$$

\noindent Mean average precision is mean for average precision of multiple queries

\subsection{Normalised Discounted Cumulative Gain}

\noindent Cumulative gain:

$$\text{CG}=r_1 + r_2 + \ldots + r_n$$

\noindent Discounted cumulative gain:
$$\text{DCG} = \frac{r_1}{log_2 1} \frac{r_2}{log_2 2} + \ldots + \frac{r_n}{log_2 n}$$

\noindent Ideal DCG (IDCG) is calculated using DCG formula, but the documents are sorted based on relevancy before the calculation is done. \\

\noindent Normalise Discounted Cumulative Gain:
$$\text{nDCG} = \frac{\text{DCG}}{\text{IDCG}}$$

\subsection{Mean Reciprocal Rank (MRR)}

\noindent Consider the rank position of first relevant document as $K$. The reciprocal rank score for the query is $1 / K$. MRR is the mean of all reciprocal rank scores for some queries. 

\section{Inter-judge Agreement and Kappa Measure}

\begin{center}
\begin{tabular}{ |c|c c| } 
    \hline
     & J1: Yes & J1:No \\
    \hline 
    J2: Yes & A & B \\
    J2: No & C & D \\
    \hline
\end{tabular}
\end{center}

$$P(\text{agree}) = \frac{A+D}{A+B+C+D}$$
$$P(\text{nonrelevant}) = \frac{C + D + B +D}{2(A+B+C+D)}$$
$$P(\text{relevant}) = \frac{A+B+A+C}{2(A+B+C+D)}$$
$$P(\text{E}) = P(\text{nonrelevant})^2 + P(\text{relevant})^2$$
$$\text{Kappa}=\frac{P(A)-P(E)}{1-P(E)}$$

\section{Relevance Feedback}

\noindent Relevance feedback: user feedback on relevance of documents in initial set of results \\

\noindent The centroid is the center of mass of a set of points:
$$\vec{\mu}(C)=\frac{1}{|C|} \sum_{d\in C} \vec{d}$$

\subsection{Rocchio Algorithm}
\noindent Try to separate documents marked as relevant and non-relevant, but we don’t know the truly relevant documents:

$$\vec{q}_{opt}=\arg\!\max_{\vec{q}}[ \cos(\vec{q},\vec{\mu}(C_r)) - \cos(\vec{q},\vec{\mu}(C_{nr}))]$$

\noindent Rocchio Algorithm:

$$\vec{q}_m = \alpha \vec{q}_o + \beta \frac{1}{|D_r|} \sum_{\vec{d}_j \in D_r} \vec{d}_j + \gamma \frac{1}{D_{nr}} \sum_{\vec{d}_j \in D_{nr}} \vec{d}_j$$

\subsection{Assumptions of Relevance Feedback}

\noindent Assumption 1: User has sufficient knowledge for initial query \\

\noindent Assumption 2: Relevance prototypes are well-behaved and there is a clear separation between relevant and non-relevant documents\\

\noindent Both assumptions are usually wrong

\subsection{Pseudo Relevance Feedback}
Automate the manual part. Algorithm:
\begin{itemize}
    \item Retrieve a ranked list for user’s query
    \item Assume that top k documents are relevant
    \item Do relevance feedback
\end{itemize}

\section{Query Expansion}

\noindent In query expansion, users give additional input on words or phrases \\

\noindent Thesaurus-based query expansion: for each term in a query, expand the query with synonyms and related words of term from thesaurus. \\

\noindent Co-occurrence thesaurus: two words are similar if they co-occur with similar words \\

\noindent Semantic thesaurus: two words are similar if they occur in a given grammatical relation with the same words

\end{multicols*}
